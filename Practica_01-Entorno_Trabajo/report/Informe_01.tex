\documentclass[11pt]{report}

% Paquetes y configuraciones adicionales
\usepackage{graphicx}
\usepackage[export]{adjustbox}
\usepackage{caption}
\usepackage{float}
\usepackage{titlesec}
\usepackage{geometry}
\usepackage[hidelinks]{hyperref}
\usepackage{titling}
\usepackage{titlesec}
\usepackage{parskip}
\usepackage{wasysym}
\usepackage{tikzsymbols}
\usepackage{fancyvrb}
\usepackage{xurl}
\usepackage{hyperref}
\usepackage{subcaption}
\usepackage{tcolorbox}
\usepackage{listings}
\usepackage{xcolor}
\usepackage{tabularx}   % en el preámbulo
\usepackage{array}      % para >{\raggedright}p{...}
\usepackage{seqsplit}  % permite partir cadenas largas dentro de la tabla


\usepackage[spanish]{babel}

\newcommand{\subtitle}[1]{
  \posttitle{
    \par\end{center}
    \begin{center}\large#1\end{center}
    \vskip0.5em}
}

% Configura los márgenes
\geometry{
  left=2cm,   % Ajusta este valor al margen izquierdo deseado
  right=2cm,  % Ajusta este valor al margen derecho deseado
  top=3cm,
  bottom=3cm,
}

% Configuración de los títulos de las secciones
\titlespacing{\section}{0pt}{\parskip}{\parskip}
\titlespacing{\subsection}{0pt}{\parskip}{\parskip}
\titlespacing{\subsubsection}{0pt}{\parskip}{\parskip}

% Redefinir el formato de los capítulos y añadir un punto después del número
\makeatletter
\renewcommand{\@makechapterhead}[1]{%
  \vspace*{0\p@} % Ajusta este valor para el espaciado deseado antes del título del capítulo
  {\parindent \z@ \raggedright \normalfont
    \ifnum \c@secnumdepth >\m@ne
        \huge\bfseries \thechapter.\ % Añade un punto después del número
    \fi
    \interlinepenalty\@M
    #1\par\nobreak
    \vspace{10pt} % Ajusta este valor para el espacio deseado después del título del capítulo
  }}
\makeatother

% Configura para que cada \chapter no comience en una pagina nueva
\makeatletter
\renewcommand\chapter{\@startsection{chapter}{0}{\z@}%
    {-3.5ex \@plus -1ex \@minus -.2ex}%
    {2.3ex \@plus.2ex}%
    {\normalfont\Large\bfseries}}
\makeatother

% Configurar los colores para el código
\definecolor{codegreen}{rgb}{0,0.6,0}
\definecolor{codegray}{rgb}{0.5,0.5,0.5}
\definecolor{codepurple}{rgb}{0.58,0,0.82}
\definecolor{backcolour}{rgb}{0.95,0.95,0.92}

% Configurar el estilo para el código
\lstdefinestyle{mystyle}{
  backgroundcolor=\color{backcolour},   
  commentstyle=\color{codegreen},
  keywordstyle=\color{magenta},
  numberstyle=\tiny\color{codegray},
  stringstyle=\color{codepurple},
  basicstyle=\ttfamily\footnotesize,
  breakatwhitespace=false,         
  breaklines=true,                 
  captionpos=b,                    
  keepspaces=true,                 
  numbers=left,                    
  numbersep=5pt,                  
  showspaces=false,                
  showstringspaces=false,
  showtabs=false,                  
  tabsize=2
}

% \renewcommand{\thechapter}{\alph{chapter}}

%==============================================================================
% Cosas para la documentación LateX
% % Sangría
% \setlength{\parindent}{1em}Texto

% % Quitar sangría
% \noindent

% % Punto
% \CIRCLE \ \ \textbf{Texto} \emph{algo}
% \begin{itemize}
%   \item \textbf{Negrita:} Texto
%   \item \textbf{Negrita:} Texto
% \end{itemize}

% % Introducir código
% \begin{center}
%   \begin{BVerbatim}
%     ... Código
%   \end{BVerbatim}
% \end{center}

% Poner una imagen
% \begin{figure}[H]
%   \centering
%   \includegraphics[scale=0.55]{img/}
%   \caption{Exportación de la base de datos en formato sql}
%   \label{fig:exportación de la base de datos en formato sql}
% \end{figure}

% Poner dos imágenes
% \begin{figure}[H]
%   \begin{subfigure}{0.5\textwidth}
%     \centering
%     \includegraphics[scale=0.45]{img/}
%     \caption{Texto imagen 1}
%   \end{subfigure}%
%   \begin{subfigure}{0.5\textwidth}
%     \centering
%     \includegraphics[scale=0.45]{img/}
%     \caption{Texto imagen 2}
%   \end{subfigure}
%   \caption{Texto general}
% \end{figure}

% % Poner una tabla
% \begin{table}[H]
%   \centering
%   \begin{tabular}{|c|c|c|c|}
%     \hline
%     \textbf{Campo 1} & \textbf{Campo 2} & \textbf{Campo 3} & \textbf{Campo 4} \\ \hline
%     Texto & Texto & Texto & Texto \\ \hline
%     Texto & Texto & Texto & Texto \\ \hline
%     Texto & Texto & Texto & Texto \\ \hline
%     Texto & Texto & Texto & Texto \\ \hline
%   \end{tabular}
%   \caption{Nombre de la tabla}
%   \label{tab:nombre de la tabla}
% \end{table}

% % Poner codigo de un lenguaje a partir de un archivo
% \lstset{style=mystyle}
% The next code will be directly imported from a file
% \lstinputlisting[language=Python]{code.py}

% “Texto entre comillas dobles”

%==============================================================================

\begin{document}

% Portada del informe
\begin{titlepage}
	\begin{center}
		\includegraphics[scale=0.5]{img/logo.png}
		\vspace{1cm}

		\vspace{2cm}
		\begin{Huge}
			\textbf{Práctica 1: Entorno de trabajo auditoría de redes inalámbricas}
		\end{Huge}

		\vspace{1cm}
		\begin{large}
			\textbf{Seguridad de las Comunicaciones Inalámbricas}
		\end{large}

		\vspace{4cm}
		\begin{large}
			\textbf{Autor:} Cheuk Kelly Ng Pante (alu0101364544@ull.edu.es)
		\end{large}

		\vspace{1cm}
		\begin{large}
			\textbf{Fecha:} \today
		\end{large}
	\end{center}
\end{titlepage}

\pagestyle{empty} % Desactiva la numeración de página para el índice

% Índice
\tableofcontents

% Nueva página
\cleardoublepage

\pagestyle{plain} % Vuelve a activar la numeración de página
\setcounter{page}{1} % Reinicia el contador de página a 1

% Secciones del informe
% Capitulo 1
\chapter{Análisis y Configuración de Interfaces de Red}
Ejecutar el comando para listar todas las interfaces de red:
\begin{lstlisting}[style=mystyle]
ip addr show
\end{lstlisting}
Forma abreviada: \texttt{ip a}

\section{Enumeración de Interfaces}
\begin{itemize}
    \item \textbf{Dirección MAC:} Valor \texttt{link/ether}
    \item \textbf{Direcciones IP:} \texttt{inet} (IPv4) e \texttt{inet6} (IPv6)
    \item \textbf{Dirección Broadcast:} Valor \texttt{brd}
    \item \textbf{Estado:} \texttt{STATE UP} o \texttt{STATE DOWN}
    \item \textbf{MTU:} Valor numérico \texttt{mtu}
\end{itemize}

\begin{figure}[H]
  \centering
  \includegraphics[scale=0.45]{img/Screenshot_1.png}
  \caption{Salida del comando \texttt{ip addr show}}
  \label{fig:ip_addr_show}
\end{figure}

\newpage

\chapter{Gesión del Estado de la Interfaz y Escalado de Privilegios}
\section{Intento sin privilegios}
Intentar deshabilitar la interfaz eth0:
\begin{lstlisting}[style=mystyle]
ip link set eth0 down
\end{lstlisting}
\textit{Resultado esperado:} Error de permiso denegado.

\begin{figure}[H]
  \centering
  \includegraphics[scale=0.6]{img/Screenshot_2.png}
  \caption{Error al intentar deshabilitar eth0 sin privilegios}
  \label{fig:error_sin_privilegios}
\end{figure}

\section{Ejecución con privilegios}
Ejecutar con \texttt{sudo}:
\begin{lstlisting}[style=mystyle]
sudo ip link set eth0 down
\end{lstlisting}

\begin{figure}[H]
  \centering
  \includegraphics[scale=0.6]{img/Screenshot_3.png}
  \caption{Deshabilitación exitosa de eth0 con privilegios}
  \label{fig:deshabilitacion_exitosa}
\end{figure}

\newpage

\section{Verificación de conectividad}
Confirmar que la interfaz está inactiva:
\begin{lstlisting}[style=mystyle]
ip a
\end{lstlisting}

\begin{figure}[H]
  \centering
  \includegraphics[scale=0.35]{img/Screenshot_4.png}
  \caption{Verificación de estado inactivo de eth0}
  \label{fig:verificacion_inactivo}
\end{figure}

Al acceder a alguna página web podemos ver que se ha perdido la conexion:
\begin{figure}[H]
  \centering
  \includegraphics[scale=0.3]{img/Screenshot_4_1.png}
  \caption{Perdida de conexion}
\end{figure}

\section{Reactivación}
Restaurar la interfaz:
\begin{lstlisting}[style=mystyle]
sudo ip link set eth0 up
\end{lstlisting}
Validar la restauración con \texttt{ip a} y verificar conectividad a Internet.

\begin{figure}[H]
  \centering
  \includegraphics[scale=0.45]{img/Screenshot_5.png}
  \caption{Reactivación exitosa de eth0}
  \label{fig:reactivacion_exitosa}
\end{figure}

\subsection{Comandos adicionales}
\begin{lstlisting}[style=mystyle]
sudo ip addr add 192.168.1.200/24 dev eth0
\end{lstlisting}

\begin{figure}[H]
    \centering
    \includegraphics[scale=0.45]{img/Screenshot_6.png}
    \caption{Asignación temporal de nueva IP a eth0}
    \label{fig:nueva_ip_eth0}
\end{figure}

Al hacer esto el cambio es temporal y se pierde al reiniciar. Para hacerlo persistente hay que editar el fichero \texttt{/etc/netplan/01-network-manager-all.yaml}.

\begin{figure}[H]
    \centering
    \includegraphics[scale=0.45]{img/Screenshot_7.png}
    \caption{Verificación de nueva IP asignada a eth0}
    \label{fig:verificacion_nueva_ip}
\end{figure}

\chapter{Verificación de la Pila TCP/IP y Conectividad Externa}
\section{Prueba de Loopback}
Ping a la interfaz de loopback:
\begin{lstlisting}[style=mystyle]
ping -c 4 127.0.0.1
\end{lstlisting}

\begin{figure}[H]
  \centering
  \includegraphics[scale=0.55]{img/Screenshot_8.png}
  \caption{Prueba de conectividad a loopback}
  \label{fig:ping_loopback}
\end{figure}

\newpage

\section{Prueba de Conectividad Externa}
Verificar resolución DNS y conectividad WAN:
\begin{lstlisting}[style=mystyle]
ping -c 4 www.ull.es
\end{lstlisting}

\begin{figure}[H]
  \centering
  \includegraphics[scale=0.55]{img/Screenshot_9.png}
  \caption{Prueba de conectividad a www.ull.es}
  \label{fig:ping_ull}
\end{figure}

A continuación se documentan los resultados obtenidos tras ejecutar el comando de diagnóstico de red hacia el dominio de la Universidad de La Laguna.

\paragraph{a. Estadísticas de Paquetes}
Basado en la línea final: \textit{4 packets transmitted, 0 received, 100\% packet loss}.

\begin{itemize}
    \item \textbf{Transmitidos:} 4
    \item \textbf{Recibidos:} 0
    \item \textbf{Pérdida (\%):} 100\%
\end{itemize}

\paragraph{b. Estadísticas de RTT (Round-Trip Time)}
Debido a que la pérdida de paquetes fue total (ningún paquete retornó), el sistema no pudo calcular los tiempos de viaje.

\begin{itemize}
    \item \textbf{Mínimo:} N/A (No disponible)
    \item \textbf{Promedio (Avg):} N/A (No disponible)
    \item \textbf{Máximo:} N/A (No disponible)
\end{itemize}

\vspace{0.5cm}
\noindent \textbf{Observación:} El fallo en la recepción de paquetes (100\% de pérdida) sugiere que el host destino (\texttt{193.145.100.5}) está inactivo o, lo más probable, que existe un firewall bloqueando las solicitudes ICMP.


\chapter{Análisis de la Caché del Protocolo ARP}
Al realizar la práctica en Ubuntu 22.04, se observó que el comando clásico arp -a no está disponible por defecto, ya que pertenece al paquete obsoleto net-tools. Por este motivo, se utiliza la suite moderna iproute2, donde el comando ip neigh (abreviatura de ip neighbor) permite consultar y gestionar la tabla de vecinos, que para IPv4 corresponde a la caché ARP.

En consecuencia, en este informe se emplea el comando ip neigh show como alternativa directa a arp -a para inspeccionar el mapeo entre direcciones IP y direcciones MAC de los dispositivos de la red, incluida la puerta de enlace.

\newpage

El comando utilizado es:
\begin{lstlisting}[style=mystyle]
ip neigh
\end{lstlisting}

\begin{figure}[H]
  \centering
  \includegraphics[scale=0.55]{img/Screenshot_10.png}
  \caption{Salida del comando \texttt{ip neigh}}
  \label{fig:ip_neigh}
\end{figure}

\chapter{Inspección de la Tabla de Enrutamiento}
Analizar cómo el sistema decide dónde enviar el tráfico. Para mostrar la tabla de enrutamiento:
\begin{lstlisting}[style=mystyle]
ip route show
\end{lstlisting}
Forma abreviada: \texttt{ip r}

En la salida del comando ip route show se observan dos rutas por defecto configuradas en la interfaz \texttt{enp27s0}. Estas rutas apuntan a dos puertas de enlace distintas, \texttt{192.168.1.1} (configurada de forma estática) y \texttt{192.168.0.1} (obtenida por DHCP), ambas con la misma métrica (20100), por lo que el kernel puede utilizar cualquiera de ellas como \emph{default gateway} en función de su disponibilidad.

\begin{figure}[H]
  \centering
  \includegraphics[scale=0.55]{img/Screenshot_11.png}
  \caption{Salida del comando \texttt{ip route show}}
  \label{fig:ip_route_show}
\end{figure}

\chapter{Trazado de Ruta de Red (Traceroute)}
Los trazados de ruta realizados son:
\begin{lstlisting}[style=mystyle]
traceroute www.ull.es
traceroute www.net.berkeley.edu
\end{lstlisting}

% \begin{figure}[H]
%   \centering
%   \includegraphics[scale=0.55]{img/Screenshot_12.png}
%   \caption{Trazado de ruta a www.ull.es}
%   \label{fig:traceroute_ull}
% \end{figure}

\begin{figure}[H]
  \begin{subfigure}{0.5\textwidth}
    \centering
    \includegraphics[scale=0.35]{img/Screenshot_12_1.png}
    \caption{Trazado de ruta a www.ull.es}
  \end{subfigure}%
  \begin{subfigure}{0.5\textwidth}
    \centering
    \includegraphics[scale=0.35]{img/Screenshot_12_2.png}
    \caption{Trazado de ruta a www.net.berkeley.edu}
  \end{subfigure}
  \caption{Trazados de ruta a dos destinos diferentes}
\end{figure}

Al realizar la práctica en Ubuntu 22.04 se ejecutó el comando \texttt{traceroute} hacia dos destinos: \texttt{www.net.berkeley.edu} y \texttt{www.ull.es}, con el objetivo de identificar los routers intermedios y la latencia asociada a cada salto de Capa~3. En el caso de \texttt{www.net.berkeley.edu}, el trazado muestra un total de 14 saltos desde la puerta de enlace local (\texttt{192.168.0.1}), con un tiempo de ida y vuelta inicial de aproximadamente 3~ms y latencias que aumentan progresivamente hasta valores cercanos a 170--180~ms en el destino final, lo que refleja la distancia geográfica y el número de dominios de red atravesados.

Para \texttt{www.ull.es}, el traceroute alcanza la puerta de enlace local y varios routers pertenecientes a la red académica (por ejemplo, nodos de RedIRIS), con latencias en torno a 30--60~ms. A partir de cierto punto, numerosos saltos aparecen como \texttt{* * *}, lo que indica que esos routers no responden a las sondas de \texttt{traceroute} (por filtrado o limitación de tráfico ICMP), aun cuando la conectividad con el destino sigue siendo funcional.

\chapter{Escaneo de Puertos Locales con Nmap}
Para realizar el auto-descubrimiento de servicios se ejecutó un escaneo con \texttt{nmap} sobre la propia máquina (\texttt{nmap -sV localhost}). Este comando permite identificar qué puertos TCP se encuentran en estado \emph{LISTEN} y, además, intenta determinar la versión de los servicios asociados mediante el flag \texttt{-sV}, lo que simula un reconocimiento inicial desde la perspectiva de un atacante situado en la misma red local.

\begin{figure}[H]
  \centering
  \includegraphics[scale=0.45]{img/Screenshot_13.png}
  \caption{Salida del comando \texttt{nmap -sV localhost}}
  \label{fig:nmap_localhost}
\end{figure}

Al ejecutar el comando \texttt{nmap -sV localhost} sobre la propia máquina, se comprobó que el host \texttt{localhost} (\texttt{127.0.0.1}) está activo y responde con una latencia muy baja. El escaneo analiza los 1000 puertos TCP más habituales y muestra que 999 de ellos se encuentran cerrados, detectándose únicamente un puerto abierto: el \texttt{631/tcp}, asociado al servicio \texttt{ipp} (Internet Printing Protocol), que en este caso corresponde a \texttt{CUPS 2.4}, es decir, el subsistema de impresión del sistema operativo escuchando en la interfaz de \emph{loopback}.

\chapter{Análisis de Sockets y Conexiones de Red}
Para inspeccionar los sockets y conexiones de red activas en el sistema, se utilizó el comando \texttt{ss} (socket statistics), que forma parte de la suite \texttt{iproute2} y ofrece una visión detallada de las conexiones TCP, UDP y otros tipos de sockets.
\begin{lstlisting}[style=mystyle]
ss -tuln
\end{lstlisting}

\begin{figure}[H]
  \centering
  \includegraphics[scale=0.45]{img/Screenshot_14.png}
  \caption{Salida del comando \texttt{ss -tuln}}
  \label{fig:ss_tuln}
\end{figure}

Al ejecutar el comando \texttt{ss -tuln} se obtuvo un listado de los sockets UDP y TCP abiertos en el sistema, incluyendo su estado y la dirección/puerto local asociados. En la parte UDP, los sockets aparecen en estado \texttt{UNCONN}, con varios puertos \texttt{53} y \texttt{5353} escuchando en direcciones como \texttt{127.0.0.53\%lo}, \texttt{192.168.122.1} y \texttt{0.0.0.0}, lo que indica la presencia de servicios de resolución de nombres (DNS) y de descubrimiento en la red local, así como un servidor DHCP vinculado a la interfaz virtual \texttt{virbr0}.

En la sección TCP, los sockets se muestran en estado \texttt{LISTEN}, destacando que la mayoría de servicios están restringidos a la interfaz de \emph{loopback} (por ejemplo, \texttt{127.0.0.53\%lo:53}, \texttt{127.0.0.1:631} y \texttt{[::1]:631}, correspondientes a servicios de DNS local y CUPS), mientras que otros puertos como \texttt{192.168.122.1:53} sólo son accesibles desde la red virtual interna. En conjunto, la salida evidencia que los servicios expuestos hacia el exterior son mínimos y que los puertos más sensibles se limitan a la propia máquina o a redes virtuales controladas.

\chapter{Resolución de Nombres DNS (NSLookup)}

\section{Objetivo}
Realizar consultas DNS para resolver nombres de dominio a \texttt{www.ull.es} y \texttt{www.w3c.org} se utilizaron consultas DNS mediante \texttt{nslookup}. En ambos casos se identificó el servidor DNS que resolvió la consulta (línea \texttt{Server:}) y se registraron las direcciones IPv4 devueltas como registros de tipo A para cada dominio.

\begin{figure}[H]
  \centering
  \includegraphics[scale=0.7]{img/Screenshot_15.png}
  \caption{Salida del comando \texttt{nslookup <dominio>}}
  \label{fig:nslookup}
\end{figure}

En las consultas DNS realizadas con \texttt{nslookup} para \texttt{www.ull.es} y \texttt{www.w3c.org}, el servidor que resuelve ambas peticiones es \texttt{127.0.0.53}, es decir, el resolvedor local de Ubuntu que actúa como caché y reenvía las consultas a los DNS configurados en el sistema.

Para \texttt{www.ull.es}, la respuesta incluye un nombre canónico \texttt{w4.stic.ull.es} y un registro A con la dirección IPv4 \texttt{193.145.100.5}. Para \texttt{www.w3c.org}, el nombre canónico resuelto es \texttt{webredir.vip.gandi.net} y el registro A obtenido es la dirección IPv4 \texttt{217.70.184.50}

\chapter{Interacción de Red con Netcat (nc)}
Se utilizó Netcat (\texttt{nc}) como herramienta de depuración de red. En modo escucha se empleó la sintaxis \texttt{nc -l -p 1234} para iniciar un servicio que acepta conexiones entrantes en el puerto 1234. En modo cliente se utilizó \texttt{nc <host> 1234} para establecer una conexión TCP hacia dicho puerto remoto y verificar el intercambio de datos extremo a extremo.

\newpage

\section{Sintaxis clave}
\textbf{Modo Escucha (Listener):}
\begin{lstlisting}[style=mystyle]
nc -l -p 1234
\end{lstlisting}

\textbf{Modo Cliente:}
\begin{lstlisting}[style=mystyle]
nc [IP_remota] [puerto]
\end{lstlisting}


En la Figura~\ref{fig:netcat} se muestra la configuración de Netcat en modo escucha y cliente en dos terminales diferentes.

\begin{figure}[H]
  \centering
  \includegraphics[scale=0.7]{img/Screenshot_16.png}
  \caption{Uso de Netcat en modo escucha y cliente}
  \label{fig:netcat}
\end{figure}

Para probar la interacción de red con Netcat se lanzó un listener en la máquina local mediante \texttt{nc -l -p 1234} y, en una segunda terminal, un cliente con \texttt{nc 127.0.0.1 1234}. Una vez establecida la conexión, cualquier texto introducido en una de las terminales aparecía en la otra, demostrando que se había creado un canal TCP bidireccional entre cliente y servidor sobre el puerto 1234 de \texttt{localhost}, lo que simula el funcionamiento básico de un servicio y su cliente en la misma red.​

\chapter{Reconocimiento DNS Avanzado con dnsenum}
Se ejecutó \texttt{dnsenum tecnomobile.com} para realizar enumeración DNS automatizada del dominio, obteniendo direcciones de host (registros A/AAAA), servidores de nombres (NS), servidores de correo (MX) y, adicionalmente, posibles subdominios y rangos de red asociados descubiertos por fuerza bruta y consultas complementarias.

\begin{figure}[H]
  \begin{subfigure}{0.5\textwidth}
    \centering
    \includegraphics[scale=0.4]{img/Screenshot_17_1.png}
  \end{subfigure}%
  \begin{subfigure}{0.5\textwidth}
    \centering
    \includegraphics[scale=0.4]{img/Screenshot_17_2.png}
  \end{subfigure}
  \caption{Salida del comando \texttt{dnsenum tecnomobile.com}}
\end{figure}

Mediante la ejecución de \texttt{dnsenum tecnomobile.com} se identificó que el dominio principal \texttt{tecnomobile.com} dispone de un registro A asociado a la dirección IPv4 \texttt{37.48.77.81}. Asimismo, se detectaron como servidores de nombres (registros NS) los hosts \texttt{ns1.quokkadns.com} (\texttt{207.244.71.177}) y \texttt{ns2.quokkadns.com} (\texttt{5.79.65.14}). La herramienta no devolvió registros MX para el dominio, pero sí enumeró varios subdominios numéricos (por ejemplo, \texttt{1003.tecnomobile.com}, \texttt{1025.tecnomobile.com}, etc.) con distintas direcciones IPv4, lo que aporta información adicional sobre la infraestructura asociada al dominio.

\newpage

\chapter{Análisis de Tráfico en Terminal con tcpdump}
Se realizó una captura básica con \texttt{tcpdump} mediante el comando \texttt{sudo tcpdump -i enp27s0 -n -c 20}. En las primeras líneas se observa tráfico de resolución ARP dentro de la red local (petición para averiguar la MAC de \texttt{192.168.0.1} desde \texttt{192.168.0.37}), así como algún frame Ethernet de broadcast de tipo no identificado por la herramienta. A continuación aparecen múltiples paquetes IP entre la máquina local \texttt{192.168.0.200} y distintos servidores externos en el puerto \texttt{443}, tanto en UDP (probablemente tráfico QUIC) como en TCP, donde pueden verse flags de \emph{push} y \emph{finish} que indican el envío de datos de aplicaciones web y el cierre ordenado de las conexiones TLS.

\begin{figure}[H]
  \centering
  \includegraphics[scale=0.55]{img/Screenshot_18.png}
  \caption{Salida del comando \texttt{tcpdump}}
  \label{fig:tcpdump}
\end{figure}



\end{document}