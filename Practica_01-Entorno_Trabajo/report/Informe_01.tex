\documentclass[11pt]{report}

% Paquetes y configuraciones adicionales
\usepackage{graphicx}
\usepackage[export]{adjustbox}
\usepackage{caption}
\usepackage{float}
\usepackage{titlesec}
\usepackage{geometry}
\usepackage[hidelinks]{hyperref}
\usepackage{titling}
\usepackage{titlesec}
\usepackage{parskip}
\usepackage{wasysym}
\usepackage{tikzsymbols}
\usepackage{fancyvrb}
\usepackage{xurl}
\usepackage{hyperref}
\usepackage{subcaption}
\usepackage{tcolorbox}
\usepackage{listings}
\usepackage{xcolor}
\usepackage{tabularx}   % en el preámbulo
\usepackage{array}      % para >{\raggedright}p{...}
\usepackage{seqsplit}  % permite partir cadenas largas dentro de la tabla


\usepackage[spanish]{babel}

\newcommand{\subtitle}[1]{
  \posttitle{
    \par\end{center}
    \begin{center}\large#1\end{center}
    \vskip0.5em}
}

% Configura los márgenes
\geometry{
  left=2cm,   % Ajusta este valor al margen izquierdo deseado
  right=2cm,  % Ajusta este valor al margen derecho deseado
  top=3cm,
  bottom=3cm,
}

% Configuración de los títulos de las secciones
\titlespacing{\section}{0pt}{\parskip}{\parskip}
\titlespacing{\subsection}{0pt}{\parskip}{\parskip}
\titlespacing{\subsubsection}{0pt}{\parskip}{\parskip}

% Redefinir el formato de los capítulos y añadir un punto después del número
\makeatletter
\renewcommand{\@makechapterhead}[1]{%
  \vspace*{0\p@} % Ajusta este valor para el espaciado deseado antes del título del capítulo
  {\parindent \z@ \raggedright \normalfont
    \ifnum \c@secnumdepth >\m@ne
        \huge\bfseries \thechapter.\ % Añade un punto después del número
    \fi
    \interlinepenalty\@M
    #1\par\nobreak
    \vspace{10pt} % Ajusta este valor para el espacio deseado después del título del capítulo
  }}
\makeatother

% Configura para que cada \chapter no comience en una pagina nueva
\makeatletter
\renewcommand\chapter{\@startsection{chapter}{0}{\z@}%
    {-3.5ex \@plus -1ex \@minus -.2ex}%
    {2.3ex \@plus.2ex}%
    {\normalfont\Large\bfseries}}
\makeatother

% Configurar los colores para el código
\definecolor{codegreen}{rgb}{0,0.6,0}
\definecolor{codegray}{rgb}{0.5,0.5,0.5}
\definecolor{codepurple}{rgb}{0.58,0,0.82}
\definecolor{backcolour}{rgb}{0.95,0.95,0.92}

% Configurar el estilo para el código
\lstdefinestyle{mystyle}{
  backgroundcolor=\color{backcolour},   
  commentstyle=\color{codegreen},
  keywordstyle=\color{magenta},
  numberstyle=\tiny\color{codegray},
  stringstyle=\color{codepurple},
  basicstyle=\ttfamily\footnotesize,
  breakatwhitespace=false,         
  breaklines=true,                 
  captionpos=b,                    
  keepspaces=true,                 
  numbers=left,                    
  numbersep=5pt,                  
  showspaces=false,                
  showstringspaces=false,
  showtabs=false,                  
  tabsize=2
}

% \renewcommand{\thechapter}{\alph{chapter}}

%==============================================================================
% Cosas para la documentación LateX
% % Sangría
% \setlength{\parindent}{1em}Texto

% % Quitar sangría
% \noindent

% % Punto
% \CIRCLE \ \ \textbf{Texto} \emph{algo}
% \begin{itemize}
%   \item \textbf{Negrita:} Texto
%   \item \textbf{Negrita:} Texto
% \end{itemize}

% % Introducir código
% \begin{center}
%   \begin{BVerbatim}
%     ... Código
%   \end{BVerbatim}
% \end{center}

% Poner una imagen
% \begin{figure}[H]
%   \centering
%   \includegraphics[scale=0.55]{img/}
%   \caption{Exportación de la base de datos en formato sql}
%   \label{fig:exportación de la base de datos en formato sql}
% \end{figure}

% Poner dos imágenes
% \begin{figure}[H]
%   \begin{subfigure}{0.5\textwidth}
%     \centering
%     \includegraphics[scale=0.45]{img/}
%     \caption{Texto imagen 1}
%   \end{subfigure}%
%   \begin{subfigure}{0.5\textwidth}
%     \centering
%     \includegraphics[scale=0.45]{img/}
%     \caption{Texto imagen 2}
%   \end{subfigure}
%   \caption{Texto general}
% \end{figure}

% % Poner una tabla
% \begin{table}[H]
%   \centering
%   \begin{tabular}{|c|c|c|c|}
%     \hline
%     \textbf{Campo 1} & \textbf{Campo 2} & \textbf{Campo 3} & \textbf{Campo 4} \\ \hline
%     Texto & Texto & Texto & Texto \\ \hline
%     Texto & Texto & Texto & Texto \\ \hline
%     Texto & Texto & Texto & Texto \\ \hline
%     Texto & Texto & Texto & Texto \\ \hline
%   \end{tabular}
%   \caption{Nombre de la tabla}
%   \label{tab:nombre de la tabla}
% \end{table}

% % Poner codigo de un lenguaje a partir de un archivo
% \lstset{style=mystyle}
% The next code will be directly imported from a file
% \lstinputlisting[language=Python]{code.py}

% “Texto entre comillas dobles”

%==============================================================================

\begin{document}

% Portada del informe
\begin{titlepage}
	\begin{center}
		\includegraphics[scale=0.5]{img/logo.png}
		\vspace{1cm}

		\vspace{2cm}
		\begin{Huge}
			\textbf{Práctica 1: Ataque por denegación de servicios}
		\end{Huge}

		\vspace{1cm}
		\begin{large}
			\textbf{Seguridad de las comunicaciones por Internet}
		\end{large}

		\vspace{4cm}
		\begin{large}
			\textbf{Autor:} Cheuk Kelly Ng Pante (alu0101364544@ull.edu.es)
		\end{large}

		\vspace{1cm}
		\begin{large}
			\textbf{Fecha:} \today
		\end{large}
	\end{center}
\end{titlepage}

\pagestyle{empty} % Desactiva la numeración de página para el índice

% Índice
\tableofcontents

% Nueva página
\cleardoublepage

\pagestyle{plain} % Vuelve a activar la numeración de página
\setcounter{page}{1} % Reinicia el contador de página a 1

% Secciones del informe
% Capitulo 1
\chapter{Análisis y Configuración de Interfaces de Red}

\section{Objetivo}
Arrancar la máquina virtual de Kali Linux y analizar las interfaces de red utilizando comandos modernos.

\section{Desarrollo}
\subsection{Inicio de sesión}
Usuario: \texttt{kali}, Contraseña: \texttt{kali}

\subsection{Comando principal}
Ejecutar el comando para listar todas las interfaces de red:
\begin{lstlisting}
ip addr show
\end{lstlisting}
Forma abreviada: \texttt{ip a}

\subsection{Enumeración de Interfaces}
Para cada interfaz activa (eth0, lo, wlan0), documentar:

\begin{itemize}
    \item \textbf{Dirección MAC:} Valor \texttt{link/ether}
    \item \textbf{Direcciones IP:} \texttt{inet} (IPv4) e \texttt{inet6} (IPv6)
    \item \textbf{Dirección Broadcast:} Valor \texttt{brd}
    \item \textbf{Estado:} \texttt{STATE UP} o \texttt{STATE DOWN}
    \item \textbf{MTU:} Valor numérico \texttt{mtu}
\end{itemize}

\section{Resultados}
\textit{[Aquí se deben incluir las capturas de pantalla y los datos obtenidos]}


\chapter{Gesión del Estado de la Interfaz y Escalado de Privilegios}

\section{Objetivo}
Comprender cómo modificar el estado de las interfaces de red requiere privilegios elevados.

\section{Desarrollo}

\subsection{Intento sin privilegios}
Intentar deshabilitar la interfaz eth0:
\begin{lstlisting}
ip link set eth0 down
\end{lstlisting}
\textit{Resultado esperado:} Error de permiso denegado.

\subsection{Ejecución con privilegios}
Ejecutar con \texttt{sudo}:
\begin{lstlisting}
sudo ip link set eth0 down
\end{lstlisting}

\subsection{Verificación de conectividad}
Confirmar que la interfaz está inactiva:
\begin{lstlisting}
ip a
\end{lstlisting}
Intentar acceder a un sitio web para verificar pérdida de conectividad.

\subsection{Reactivación}
Restaurar la interfaz:
\begin{lstlisting}
sudo ip link set eth0 up
\end{lstlisting}
Validar la restauración con \texttt{ip a} y verificar conectividad a Internet.

\subsection{Comandos adicionales}
\begin{enumerate}
    \item \textbf{Cambiar IP de eth0:}
    \begin{lstlisting}
sudo ip addr add 192.168.1.200/24 dev eth0
    \end{lstlisting}
    
    \item \textbf{Asignar nueva máscara:} Se incluye en la notación CIDR (/24)
    
    \item \textbf{Configuración persistente:} Editar \texttt{/etc/network/interfaces}
\end{enumerate}

\section{Resultados}
\textit{[Documentar los resultados obtenidos]}


\chapter{Configuración de Direcciones IP Estáticas}

\section{Objetivo}
Comprender la diferencia entre configuración temporal y persistente.

\section{Asignación Temporal}
La configuración con \texttt{ip} es volátil:
\begin{lstlisting}
sudo ip addr add 192.168.1.200/24 dev eth0
\end{lstlisting}
Esta configuración se pierde al reiniciar.

\section{Configuración Persistente}
En sistemas Debian/Kali, editar el fichero \texttt{/etc/network/interfaces} para definir:
\begin{itemize}
    \item IP estática
    \item Máscara de red
    \item Puerta de enlace (gateway)
\end{itemize}

\textit{Nota: Este apartado es informativo y no requiere entrega.}


\chapter{Verificación de la Pila TCP/IP y Conectividad Externa}

\section{Objetivo}
Validar la correcta inicialización de la pila TCP/IP local y la conectividad externa.

\section{Desarrollo}

\subsection{Prueba de Loopback}
Ping a la interfaz de loopback:
\begin{lstlisting}
ping -c 4 127.0.0.1
\end{lstlisting}

\subsection{Prueba de Conectividad Externa}
Verificar resolución DNS y conectividad WAN:
\begin{lstlisting}
ping -c 4 www.ull.es
\end{lstlisting}

\section{Análisis de Resultados}
Documentar:
\begin{enumerate}
    \item \textbf{Estadísticas de paquetes:} Transmitidos, recibidos, pérdida
    \item \textbf{RTT (Round-Trip Time):} Mínimo, promedio, máximo
\end{enumerate}


\chapter{Análisis de la Caché del Protocolo ARP}

\section{Objetivo}
Inspeccionar la traducción de direcciones IP a MAC en la LAN.

\section{Desarrollo}

\subsection{Comando tradicional}
\begin{lstlisting}
arp -a
\end{lstlisting}

\subsection{Alternativa moderna}
Usando el comando \texttt{ip}:
\begin{lstlisting}
ip neigh
\end{lstlisting}

\section{Resultados}
Anotar las entradas que mapean IP-MAC de dispositivos en la red, especialmente la puerta de enlace (router).


\chapter{Inspección de la Tabla de Enrutamiento}

\section{Objetivo}
Analizar cómo el sistema decide dónde enviar el tráfico.

\section{Desarrollo}
Mostrar la tabla de enrutamiento:
\begin{lstlisting}
ip route show
\end{lstlisting}
Forma abreviada: \texttt{ip r}

\section{Análisis}
Identificar la ruta por defecto (\texttt{default via ...}) que indica la IP de la puerta de enlace.


\chapter{Trazado de Ruta de Red (Traceroute)}

\section{Objetivo}
Determinar los saltos (routers intermedios) entre la máquina y un destino.

\section{Desarrollo}
Realizar trazados de ruta:
\begin{lstlisting}
traceroute www.ull.es
traceroute www.net.berkeley.edu
\end{lstlisting}

\section{Resultados}
Documentar:
\begin{itemize}
    \item Número total de saltos para cada destino
    \item Latencia en cada nodo
\end{itemize}


\chapter{Escaneo de Puertos Locales con Nmap}

\section{Objetivo}
Identificar puertos que están escuchando y servicios asociados en la máquina local.

\section{Desarrollo}
Escaneo con detección de versión:
\begin{lstlisting}
nmap -sV localhost
\end{lstlisting}

El flag \texttt{-sV} intenta determinar la versión del servicio.

\section{Análisis}
Esto simula un reconocimiento inicial desde la perspectiva de un atacante en la misma red.


\chapter{Análisis de Sockets y Conexiones de Red}

\section{Objetivo}
Enumerar conexiones activas y puertos abiertos.

\section{Desarrollo}

\subsection{Comando tradicional}
\begin{lstlisting}
netstat -tulpn
\end{lstlisting}

\subsection{Herramienta moderna}
Usando \texttt{ss} (socket statistics):
\begin{lstlisting}
ss -tulpn
\end{lstlisting}

\section{Análisis}
Buscar puertos abiertos inesperados o conexiones remotas \texttt{ESTABLISHED} sospechosas.


\chapter{Resolución de Nombres DNS (NSLookup)}

\section{Objetivo}
Realizar consultas DNS para resolver nombres de dominio.

\section{Desarrollo}
Consultas con \texttt{nslookup}:
\begin{lstlisting}
nslookup www.ull.es
nslookup www.w3c.org
\end{lstlisting}

\section{Documentación}
Identificar y documentar:
\begin{itemize}
    \item Servidor DNS que resuelve la consulta (indicado como \texttt{Server})
    \item Registros A (IPv4) resueltos para cada dominio
\end{itemize}


\chapter{Interacción de Red con Netcat (nc)}

\section{Objetivo}
Comprender el uso de Netcat para depuración y explotación de redes.

\section{Desarrollo}

\subsection{Revisión de opciones}
\begin{lstlisting}
nc -h
\end{lstlisting}

\subsection{Sintaxis clave}
\begin{enumerate}
    \item \textbf{Modo Escucha (Listener):}
    \begin{lstlisting}
nc -l -p 1234
    \end{lstlisting}
    
    \item \textbf{Modo Cliente:}
    \begin{lstlisting}
nc <IP_remota> <puerto>
    \end{lstlisting}
\end{enumerate}


\chapter{Reconocimiento DNS Avanzado con dnsenum}

\section{Objetivo}
Recopilar inteligencia de fuentes abiertas (OSINT) sobre un dominio.

\section{Desarrollo}
Ejecutar \texttt{dnsenum}:
\begin{lstlisting}
dnsenum tecnomobile.com
\end{lstlisting}

\section{Análisis}
Extraer información clave:
\begin{enumerate}
    \item Registros de Host (A/AAAA)
    \item Servidores de Nombres (NS)
    \item Servidores de Correo (MX)
    \item Sub-enumeración de subdominios
\end{enumerate}


\chapter{Análisis de Tráfico en Terminal con tcpdump}

\section{Objetivo}
Analizar paquetes de red en línea de comandos.

\section{Desarrollo}
Captura básica (requiere privilegios):
\begin{lstlisting}
sudo tcpdump -i eth0 -n -c 20
\end{lstlisting}

\subsection{Explicación de flags}
\begin{itemize}
    \item \texttt{-i eth0}: Escucha en la interfaz eth0
    \item \texttt{-n}: No resuelve nombres DNS/IPs
    \item \texttt{-c 20}: Captura 20 paquetes y se detiene
\end{itemize}

\section{Observaciones}
Observar el flujo de tráfico en tiempo real y documentar los datos relevantes.


\end{document}