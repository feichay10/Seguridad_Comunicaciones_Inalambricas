\documentclass[11pt]{report}

% Paquetes y configuraciones adicionales
\usepackage{graphicx}
\usepackage[export]{adjustbox}
\usepackage{caption}
\usepackage{float}
\usepackage{titlesec}
\usepackage{geometry}
\usepackage[hidelinks]{hyperref}
\usepackage{titling}
\usepackage{titlesec}
\usepackage{parskip}
\usepackage{wasysym}
\usepackage{tikzsymbols}
\usepackage{fancyvrb}
\usepackage{xurl}
\usepackage{hyperref}
\usepackage{subcaption}
\usepackage{tcolorbox}
\usepackage{listings}
\usepackage{xcolor}
\usepackage{tabularx}   % en el preámbulo
\usepackage{array}      % para >{\raggedright}p{...}
\usepackage{seqsplit}  % permite partir cadenas largas dentro de la tabla


\usepackage[spanish]{babel}

\newcommand{\subtitle}[1]{
  \posttitle{
    \par\end{center}
    \begin{center}\large#1\end{center}
    \vskip0.5em}
}

% Configura los márgenes
\geometry{
  left=2cm,   % Ajusta este valor al margen izquierdo deseado
  right=2cm,  % Ajusta este valor al margen derecho deseado
  top=3cm,
  bottom=3cm,
}

% Configuración de los títulos de las secciones
\titlespacing{\section}{0pt}{\parskip}{\parskip}
\titlespacing{\subsection}{0pt}{\parskip}{\parskip}
\titlespacing{\subsubsection}{0pt}{\parskip}{\parskip}

% Redefinir el formato de los capítulos y añadir un punto después del número
\makeatletter
\renewcommand{\@makechapterhead}[1]{%
  \vspace*{0\p@} % Ajusta este valor para el espaciado deseado antes del título del capítulo
  {\parindent \z@ \raggedright \normalfont
    \ifnum \c@secnumdepth >\m@ne
        \huge\bfseries \thechapter.\ % Añade un punto después del número
    \fi
    \interlinepenalty\@M
    #1\par\nobreak
    \vspace{10pt} % Ajusta este valor para el espacio deseado después del título del capítulo
  }}
\makeatother

% Configura para que cada \chapter no comience en una pagina nueva
\makeatletter
\renewcommand\chapter{\@startsection{chapter}{0}{\z@}%
    {-3.5ex \@plus -1ex \@minus -.2ex}%
    {2.3ex \@plus.2ex}%
    {\normalfont\Large\bfseries}}
\makeatother

% Configurar los colores para el código
\definecolor{codegreen}{rgb}{0,0.6,0}
\definecolor{codegray}{rgb}{0.5,0.5,0.5}
\definecolor{codepurple}{rgb}{0.58,0,0.82}
\definecolor{backcolour}{rgb}{0.95,0.95,0.92}

% Configurar el estilo para el código
\lstdefinestyle{mystyle}{
  backgroundcolor=\color{backcolour},   
  commentstyle=\color{codegreen},
  keywordstyle=\color{magenta},
  numberstyle=\tiny\color{codegray},
  stringstyle=\color{codepurple},
  basicstyle=\ttfamily\footnotesize,
  breakatwhitespace=false,         
  breaklines=true,                 
  captionpos=b,                    
  keepspaces=true,                 
  numbers=left,                    
  numbersep=5pt,                  
  showspaces=false,                
  showstringspaces=false,
  showtabs=false,                  
  tabsize=2
}

% \renewcommand{\thechapter}{\alph{chapter}}

%==============================================================================
% Cosas para la documentación LateX
% % Sangría
% \setlength{\parindent}{1em}Texto

% % Quitar sangría
% \noindent

% % Punto
% \CIRCLE \ \ \textbf{Texto} \emph{algo}
% \begin{itemize}
%   \item \textbf{Negrita:} Texto
%   \item \textbf{Negrita:} Texto
% \end{itemize}

% % Introducir código
% \begin{center}
%   \begin{BVerbatim}
%     ... Código
%   \end{BVerbatim}
% \end{center}

% Poner una imagen
% \begin{figure}[H]
%   \centering
%   \includegraphics[scale=0.55]{img/}
%   \caption{Exportación de la base de datos en formato sql}
%   \label{fig:exportación de la base de datos en formato sql}
% \end{figure}

% Poner dos imágenes
% \begin{figure}[H]
%   \begin{subfigure}{0.5\textwidth}
%     \centering
%     \includegraphics[scale=0.45]{img/}
%     \caption{Texto imagen 1}
%   \end{subfigure}%
%   \begin{subfigure}{0.5\textwidth}
%     \centering
%     \includegraphics[scale=0.45]{img/}
%     \caption{Texto imagen 2}
%   \end{subfigure}
%   \caption{Texto general}
% \end{figure}

% % Poner una tabla
% \begin{table}[H]
%   \centering
%   \begin{tabular}{|c|c|c|c|}
%     \hline
%     \textbf{Campo 1} & \textbf{Campo 2} & \textbf{Campo 3} & \textbf{Campo 4} \\ \hline
%     Texto & Texto & Texto & Texto \\ \hline
%     Texto & Texto & Texto & Texto \\ \hline
%     Texto & Texto & Texto & Texto \\ \hline
%     Texto & Texto & Texto & Texto \\ \hline
%   \end{tabular}
%   \caption{Nombre de la tabla}
%   \label{tab:nombre de la tabla}
% \end{table}

% % Poner codigo de un lenguaje a partir de un archivo
% \lstset{style=mystyle}
% The next code will be directly imported from a file
% \lstinputlisting[language=Python]{code.py}

% “Texto entre comillas dobles”

%==============================================================================

\begin{document}

% Portada del informe
\begin{titlepage}
	\begin{center}
		\includegraphics[scale=0.5]{img/logo.png}
		\vspace{1cm}

		\vspace{2cm}
		\begin{Huge}
			\textbf{Práctica 2: Análisis ético de la exposición IoT mediante Shodan}
		\end{Huge}

		\vspace{1cm}
		\begin{large}
			\textbf{Seguridad de las comunicaciones Inalambricas}
		\end{large}

		\vspace{4cm}
		\begin{large}
			\textbf{Autor:} Cheuk Kelly Ng Pante (alu0101364544@ull.edu.es)
		\end{large}

		\vspace{1cm}
		\begin{large}
			\textbf{Fecha:} \today
		\end{large}
	\end{center}
\end{titlepage}

\pagestyle{empty} % Desactiva la numeración de página para el índice

% Índice
\tableofcontents

% Nueva página
\cleardoublepage

\pagestyle{plain} % Vuelve a activar la numeración de página
\setcounter{page}{1} % Reinicia el contador de página a 1

% Secciones del informe
% Capitulo 1
\chapter{Realizar la búsqueda referente al protocolo Modbus}
Para este caso se va a usar \textbf{port:502 modbus} en el motor de búsqueda de Shodan.\\
A continuación, se muestran algunos de los resultados obtenidos:
\begin{figure}[H]
  \centering
  \includegraphics[scale=0.33]{img/Screenshot_1.png}
  \caption{Resultado de la búsqueda Modbus en Shodan}
  \label{fig:resultado de la búsqueda Modbus en Shodan}
\end{figure}

Al seleccionar uno de los resultados, se puede observar la siguiente información:
\begin{figure}[H]
  \centering
  \includegraphics[scale=0.33]{img/Screenshot_2.png}
  \caption{Información de un dispositivo Modbus}
  \label{fig:información de un dispositivo Modbus}
\end{figure}

Y podemos ver que el puerto 80 está abierto, por lo que podemos intentar acceder a la interfaz web del dispositivo, viendo así una interfaz de login:
\begin{figure}[H]
  \centering
  \includegraphics[scale=0.33]{img/Screenshot_3_2.png}
  \caption{Interfaz web de un dispositivo Modbus}
  \label{fig:interfaz web de un dispositivo Modbus}
\end{figure}

La interfaz parece corresponder al panel de administración de un router o CPE (dispositivo de cliente provisto por un ISP), es decir, probablemente la interfaz web de gestión de un router doméstico.

\chapter{Búsqueda de dispositivos en España que usen el puerrto 47808}
Para este caso se va a usar \textbf{port:47808 country:ES} en el motor de búsqueda de Shodan.\\
A continuación, se muestran algunos de los resultados obtenidos:
\begin{figure}[H]
  \centering
  \includegraphics[scale=0.33]{img/Screenshot_4.png}
  \caption{Resultado de la búsqueda en España en Shodan}
  \label{fig:resultado de la búsqueda en España en Shodan}
\end{figure}

Al seleccionar uno de los resultados, se puede observar la siguiente información:
\begin{figure}[H]
  \centering
  \includegraphics[scale=0.33]{img/Screenshot_5.png}
  \caption{Información de un dispositivo en España}
  \label{fig:información de un dispositivo en España}
\end{figure}

Para este caso, el puerto 80 también está abierto, por lo que podemos intentar acceder a la interfaz web del dispositivo, viendo así una interfaz de login de un servidor Apache ubicado en Madrid, España:
\begin{figure}[H]
  \centering
  \includegraphics[scale=0.33]{img/Screenshot_6.png}
  \caption{Interfaz web de un dispositivo en España}
  \label{fig:interfaz web de un dispositivo en España}
\end{figure}

\chapter{Selección de 3 dispositivos y análisis de sus vulnerabilidades}
\renewcommand{\arraystretch}{1.3}

\paragraph{Dispositivos webcamxp}

\noindent

% --- Tabla resumen webcamxp ---
\noindent
\begin{tabular}{|p{4.5cm}|p{10cm}|}
    \hline
    Descripción & Cámaras IP de videovigilancia gestionadas mediante el software \texttt{webcamxp}, accesibles a través de una interfaz web HTTP/HTTPS. \\
    \hline
    Parámetros de búsqueda & \texttt{webcamxp country:es} \\
    \hline
    \# de resultados & \textit{5 resultados para \texttt{webcamxp country:es}} \\
    \hline
    Parámetros de acceso por defecto & Acceso a través de navegador web apuntando a la IP pública y puerto HTTP/HTTPS publicados; si el administrador no ha cambiado la configuración, el servicio puede estar expuesto con credenciales por defecto o incluso sin autenticación sólida. \\
    \hline
\end{tabular}

\vspace{0.4cm}

% --- Tabla detalle host 81.22.234.236 ---
\noindent
\begin{tabular}{|p{4.5cm}|p{10cm}|}
    \hline
    IP: & 81.22.234.236 \\
    \hline
    Organización: & AVATEL TELECOM, SA \\
    \hline
    Localización: & Torrevieja, Spain \\
    \hline
    Puerto: & 80/tcp (redirección a HTTPS en el mismo host) \\
    \hline
    URL: & \texttt{http://81.22.234.236/} $\rightarrow$ redirige a \texttt{https://81.22.234.236/} \\
    \hline
    Observación: & El servicio en el puerto 80 responde con un código HTTP 308 Permanent Redirect hacia HTTPS y usa el servidor web Caddy; el dispositivo está etiquetado como IoT y presenta múltiples puertos abiertos, lo que indica un sistema de videovigilancia o pasarela expuesta directamente a Internet. \\[2.5cm]
    \hline
    Evidencias: & Captura de pantalla del panel de Shodan donde se observa la IP 81.22.234.236, la organización AVATEL TELECOM, SA, la localización en Torrevieja (Spain), la lista de puertos abiertos (80, 82, 83, 84, 85, 88, 89, 90, 91, 92, 94, 97, 98, 100, 106, 111, 554) y el banner HTTP del puerto 80 con respuesta 308 y servidor Caddy. \\[2.5cm]
    \hline
    Observación & Desde el punto de vista de seguridad, la exposición de un sistema IoT con numerosos puertos abiertos y acceso web directo desde Internet incrementa la superficie de ataque; si además el servicio corresponde a un servidor de cámaras webcamxp, un atacante podría intentar localizar credenciales débiles para acceder a las imágenes de videovigilancia o a la consola de administración. \\[2.5cm]
    \hline
\end{tabular}

\paragraph{ExacqVision}

\noindent

% --- Tabla resumen ExacqVision ---
\noindent
\begin{tabular}{|p{4.5cm}|p{10cm}|}
    \hline
    Descripción & Sistemas de videovigilancia/NVR corporativos \texttt{ExacqVision}, utilizados para la grabación y gestión centralizada de cámaras IP a través de servicios web y acceso remoto. \\
    \hline
    Parámetros de búsqueda & \texttt{ExacqVision} \\
    \hline
    \# de resultados & \textit{4 resultados para \texttt{ExacqVision}} \\
    \hline
    Parámetros de acceso por defecto & Acceso mediante servicios expuestos en Internet, típicamente un servidor web IIS en el puerto 80/tcp y acceso remoto al servidor Windows mediante RDP en el puerto 3389/tcp, donde se aloja el software de videovigilancia ExacqVision. \\
    \hline
\end{tabular}

\vspace{0.4cm}

% --- Tabla detalle host 13.91.106.128 ---
\noindent
\begin{tabular}{|p{4.5cm}|p{10cm}|}
    \hline
    IP: & 13.91.106.128 \\
    \hline
    Organización: & Microsoft Corporation (AzureCloud) \\
    \hline
    Localización: & San Jose, United States (región WestUS de Azure) \\
    \hline
    Puerto: & 80/tcp (Microsoft IIS) y 3389/tcp (Remote Desktop Protocol) \\
    \hline
    URL: & \texttt{http://13.91.106.128/} (servidor web IIS sobre Windows Server 2012 R2; error 500 según el banner) \\
    \hline
    Observación: & El host es una máquina virtual en Azure con sistema operativo Windows Server 2012 R2 que ejecuta IIS en el puerto 80/tcp y expone RDP en el puerto 3389/tcp; el banner de RDP identifica el dominio y nombre de equipo \texttt{EXACQVISION}, lo que indica que se trata de un servidor asociado a la plataforma de videovigilancia ExacqVision. \\[2.5cm]
    \hline
    Evidencias: & Captura de pantalla de Shodan donde se observa la IP 13.91.106.128, la organización Microsoft Corporation (AzureCloud), la información de sistema operativo Windows Server 2012 R2, el servicio Microsoft IIS httpd en el puerto 80/tcp con código 500 Internal Server Error y el banner RDP del puerto 3389/tcp mostrando el nombre de host y dominio relacionados con ExacqVision. \\[2.5cm]
    \hline
    Observación & Desde el punto de vista de seguridad, el servidor de videovigilancia está expuesto a Internet con RDP abierto y con un sistema operativo antiguo (Windows Server 2012 R2), lo que incrementa la superficie de ataque; un atacante podría intentar explotación de vulnerabilidades de RDP o IIS, así como ataques de fuerza bruta sobre credenciales de acceso remoto para comprometer la infraestructura de videovigilancia. \\[2.5cm]
    \hline
\end{tabular}

\newpage

\paragraph{JUNG KNX}

\noindent

% --- Tabla resumen JUNG KNX ---
\noindent
\begin{tabular}{|p{4.5cm}|p{10cm}|}
    \hline
    Descripción & Pasarela/dispositivo de automatización de edificios basado en tecnología \texttt{JUNG KNX}, que actúa como KNX/IP Router para integrar el bus KNX con redes IP y permitir el control remoto de funciones domóticas (iluminación, clima, etc.). \\
    \hline
    Parámetros de búsqueda & \texttt{"JUNG" "KNX"} \quad (por ejemplo, filtrando dispositivos cuyo banner del puerto 3671/UDP se identifica como \texttt{IP Router JUNG}). \\
    \hline
    \# de resultados & \textit{4 resultados para \texttt{JUNG KNX}} \\
    \hline
    Parámetros de acceso por defecto & Acceso KNXnet/IP en puerto 3671/UDP e interfaz web en puertos HTTP (8080/tcp); autenticación básica con credenciales por defecto permite gestionar automatización desde Internet. \\
    \hline
\end{tabular}

\vspace{0.4cm}

% --- Tabla detalle host 88.19.44.142 ---
\noindent
\begin{tabular}{|p{4.5cm}|p{10cm}|}
    \hline
    IP: & 88.19.44.142 \\
    \hline
    Organización: & Telefonica de Espana SAU Red de servicios IP Spain \\
    \hline
    Localización: & Madrid, Spain \\
    \hline
    Puerto: & 3671/UDP (KNX Gateway), 8080/tcp, 8081/tcp, 8083/tcp, 8123/tcp \\
    \hline
    URL: & \texttt{http://88.19.44.142:8080/} (servidor \texttt{thttpd/2.19}; responde \texttt{401 Unauthorized} con autenticación básica “MOBOTIX Camera User”). \\
    \hline
    Observación: & El puerto 3671/UDP se identifica como \texttt{KNX Gateway} con nombre de dispositivo \texttt{IP Router JUNG}; el banner muestra información de Siemens AG y servicios soportados, confirmando que es una pasarela KNX/IP JUNG en un entorno de automatización. \\[2.5cm]
    \hline
    Evidencias: & Captura de Shodan donde se observa el dominio \texttt{rima-tde.net}, el servidor web \texttt{thttpd} en los puertos 8080/8081/8083/8123, el error HTTP 401 Unauthorized con cabecera \texttt{WWW-Authenticate: Basic realm="MOBOTIX Camera User"}, así como el banner detallado del puerto 3671/UDP describiendo el dispositivo como \texttt{KNX Gateway} y \texttt{IP Router JUNG}. \\[2.5cm]
    \hline
    Observación & La exposición de una pasarela KNX/IP a Internet con autenticación básica puede permitir a un atacante acceder al sistema de cámaras y automatización, especialmente si las credenciales son débiles o por defecto. \\[2.5cm]
    \hline
\end{tabular}


\end{document}