\documentclass[11pt]{report}

% Paquetes y configuraciones adicionales
\usepackage{graphicx}
\usepackage[export]{adjustbox}
\usepackage{caption}
\usepackage{float}
\usepackage{titlesec}
\usepackage{geometry}
\usepackage[hidelinks]{hyperref}
\usepackage{titling}
\usepackage{titlesec}
\usepackage{parskip}
\usepackage{wasysym}
\usepackage{tikzsymbols}
\usepackage{fancyvrb}
\usepackage{xurl}
\usepackage{hyperref}
\usepackage{subcaption}
\usepackage{tcolorbox}
\usepackage{listings}
\usepackage{xcolor}
\usepackage{tabularx}   % en el preámbulo
\usepackage{array}      % para >{\raggedright}p{...}
\usepackage{seqsplit}  % permite partir cadenas largas dentro de la tabla


\usepackage[spanish]{babel}

\newcommand{\subtitle}[1]{
  \posttitle{
    \par\end{center}
    \begin{center}\large#1\end{center}
    \vskip0.5em}
}

% Configura los márgenes
\geometry{
  left=2cm,   % Ajusta este valor al margen izquierdo deseado
  right=2cm,  % Ajusta este valor al margen derecho deseado
  top=3cm,
  bottom=3cm,
}

% Configuración de los títulos de las secciones
\titlespacing{\section}{0pt}{\parskip}{\parskip}
\titlespacing{\subsection}{0pt}{\parskip}{\parskip}
\titlespacing{\subsubsection}{0pt}{\parskip}{\parskip}

% Redefinir el formato de los capítulos y añadir un punto después del número
\makeatletter
\renewcommand{\@makechapterhead}[1]{%
  \vspace*{0\p@} % Ajusta este valor para el espaciado deseado antes del título del capítulo
  {\parindent \z@ \raggedright \normalfont
    \ifnum \c@secnumdepth >\m@ne
        \huge\bfseries \thechapter.\ % Añade un punto después del número
    \fi
    \interlinepenalty\@M
    #1\par\nobreak
    \vspace{10pt} % Ajusta este valor para el espacio deseado después del título del capítulo
  }}
\makeatother

% Configura para que cada \chapter no comience en una pagina nueva
\makeatletter
\renewcommand\chapter{\@startsection{chapter}{0}{\z@}%
    {-3.5ex \@plus -1ex \@minus -.2ex}%
    {2.3ex \@plus.2ex}%
    {\normalfont\Large\bfseries}}
\makeatother

% Configurar los colores para el código
\definecolor{codegreen}{rgb}{0,0.6,0}
\definecolor{codegray}{rgb}{0.5,0.5,0.5}
\definecolor{codepurple}{rgb}{0.58,0,0.82}
\definecolor{backcolour}{rgb}{0.95,0.95,0.92}

% Configurar el estilo para el código
\lstdefinestyle{mystyle}{
  backgroundcolor=\color{backcolour},   
  commentstyle=\color{codegreen},
  keywordstyle=\color{magenta},
  numberstyle=\tiny\color{codegray},
  stringstyle=\color{codepurple},
  basicstyle=\ttfamily\footnotesize,
  breakatwhitespace=false,         
  breaklines=true,                 
  captionpos=b,                    
  keepspaces=true,                 
  numbers=left,                    
  numbersep=5pt,                  
  showspaces=false,                
  showstringspaces=false,
  showtabs=false,                  
  tabsize=2
}

% \renewcommand{\thechapter}{\alph{chapter}}

%==============================================================================
% Cosas para la documentación LateX
% % Sangría
% \setlength{\parindent}{1em}Texto

% % Quitar sangría
% \noindent

% % Punto
% \CIRCLE \ \ \textbf{Texto} \emph{algo}
% \begin{itemize}
%   \item \textbf{Negrita:} Texto
%   \item \textbf{Negrita:} Texto
% \end{itemize}

% % Introducir código
% \begin{center}
%   \begin{BVerbatim}
%     ... Código
%   \end{BVerbatim}
% \end{center}

% Poner una imagen
% \begin{figure}[H]
%   \centering
%   \includegraphics[scale=0.55]{img/}
%   \caption{Exportación de la base de datos en formato sql}
%   \label{fig:exportación de la base de datos en formato sql}
% \end{figure}

% Poner dos imágenes
% \begin{figure}[H]
%   \begin{subfigure}{0.5\textwidth}
%     \centering
%     \includegraphics[scale=0.45]{img/}
%     \caption{Texto imagen 1}
%   \end{subfigure}%
%   \begin{subfigure}{0.5\textwidth}
%     \centering
%     \includegraphics[scale=0.45]{img/}
%     \caption{Texto imagen 2}
%   \end{subfigure}
%   \caption{Texto general}
% \end{figure}

% % Poner una tabla
% \begin{table}[H]
%   \centering
%   \begin{tabular}{|c|c|c|c|}
%     \hline
%     \textbf{Campo 1} & \textbf{Campo 2} & \textbf{Campo 3} & \textbf{Campo 4} \\ \hline
%     Texto & Texto & Texto & Texto \\ \hline
%     Texto & Texto & Texto & Texto \\ \hline
%     Texto & Texto & Texto & Texto \\ \hline
%     Texto & Texto & Texto & Texto \\ \hline
%   \end{tabular}
%   \caption{Nombre de la tabla}
%   \label{tab:nombre de la tabla}
% \end{table}

% % Poner codigo de un lenguaje a partir de un archivo
% \lstset{style=mystyle}
% The next code will be directly imported from a file
% \lstinputlisting[language=Python]{code.py}

% “Texto entre comillas dobles”

%==============================================================================

\begin{document}

% Portada del informe
\begin{titlepage}
	\begin{center}
		\includegraphics[scale=0.5]{img/logo.png}
		\vspace{1cm}

		\vspace{2cm}
		\begin{Huge}
			\textbf{Práctica 2: Análisis ético de la exposición IoT mediante Shodan}
		\end{Huge}

		\vspace{1cm}
		\begin{large}
			\textbf{Seguridad de las comunicaciones Inalambricas}
		\end{large}

		\vspace{4cm}
		\begin{large}
			\textbf{Autor:} Cheuk Kelly Ng Pante (alu0101364544@ull.edu.es)
		\end{large}

		\vspace{1cm}
		\begin{large}
			\textbf{Fecha:} \today
		\end{large}
	\end{center}
\end{titlepage}

\pagestyle{empty} % Desactiva la numeración de página para el índice

% Índice
\tableofcontents

% Nueva página
\cleardoublepage

\pagestyle{plain} % Vuelve a activar la numeración de página
\setcounter{page}{1} % Reinicia el contador de página a 1

% Secciones del informe
% Capitulo 1
\chapter{Realizar la búsqueda referente al protocolo Modbus}
Para este caso se va a usar \textbf{port:502 modbus} en el motor de búsqueda de Shodan.\\
A continuación, se muestran algunos de los resultados obtenidos:
\begin{figure}[H]
  \centering
  \includegraphics[scale=0.33]{img/Screenshot_1.png}
  \caption{Resultado de la búsqueda Modbus en Shodan}
  \label{fig:resultado de la búsqueda Modbus en Shodan}
\end{figure}

Al seleccionar uno de los resultados, se puede observar la siguiente información:
\begin{figure}[H]
  \centering
  \includegraphics[scale=0.33]{img/Screenshot_2.png}
  \caption{Información de un dispositivo Modbus}
  \label{fig:información de un dispositivo Modbus}
\end{figure}

Y podemos ver que el puerto 80 está abierto, por lo que podemos intentar acceder a la interfaz web del dispositivo, viendo así una interfaz de login:
\begin{figure}[H]
  \centering
  \includegraphics[scale=0.33]{img/Screenshot_3.png}
  \caption{Interfaz web de un dispositivo Modbus}
  \label{fig:interfaz web de un dispositivo Modbus}
\end{figure}

La interfaz parece corresponder al panel de administración de un router o CPE (dispositivo de cliente provisto por un ISP), es decir, probablemente la interfaz web de gestión de un router doméstico.

\chapter{Búsqueda de dispositivos en España que usen el puerrto 47808}
Para este caso se va a usar \textbf{port:47808 country:ES} en el motor de búsqueda de Shodan.\\
A continuación, se muestran algunos de los resultados obtenidos:
\begin{figure}[H]
  \centering
  \includegraphics[scale=0.33]{img/Screenshot_4.png}
  \caption{Resultado de la búsqueda en España en Shodan}
  \label{fig:resultado de la búsqueda en España en Shodan}
\end{figure}

Al seleccionar uno de los resultados, se puede observar la siguiente información:
\begin{figure}[H]
  \centering
  \includegraphics[scale=0.33]{img/Screenshot_5.png}
  \caption{Información de un dispositivo en España}
  \label{fig:información de un dispositivo en España}
\end{figure}

Para este caso, el puerto 80 también está abierto, por lo que podemos intentar acceder a la interfaz web del dispositivo, viendo así una interfaz de login de un servidor Apache ubicado en Madrid, España:
\begin{figure}[H]
  \centering
  \includegraphics[scale=0.33]{img/Screenshot_6.png}
  \caption{Interfaz web de un dispositivo en España}
  \label{fig:interfaz web de un dispositivo en España}
\end{figure}

\chapter{Selección de 3 dispositivos y análisis de sus vulnerabilidades}
\renewcommand{\arraystretch}{1.3}

\noindent
\begin{tabular}{|p{4.5cm}|p{10cm}|}
    \hline
    Descripción & \\
    \hline
    Parámetros de búsqueda & \\
    \hline
    \# de resultados & \\
    \hline
    Parámetros de acceso por defecto & \\
    \hline
\end{tabular}

% \begin{center}
%     \textit{\textcolor{magenta}{Hallazgos}}
% \end{center}

\noindent
\begin{tabular}{|p{4.5cm}|p{10cm}|}
    \hline
    IP: & \\
    \hline
    Organización: & \\
    \hline
    Localización: & \\
    \hline
    Puerto: & \\
    \hline
    URL: & \\
    \hline
    Observación: & \\[2.5cm] % Espacio extra vertical
    \hline
    Evidencias: & \\[2.5cm] % Espacio extra vertical
    \hline
    Observación & \\[2.5cm] % Espacio extra vertical
    \hline
\end{tabular}

\end{document}